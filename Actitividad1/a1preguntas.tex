%---------Definición de paquetes base--------------
\documentclass[12pt,letterpaper]{article}
\usepackage[utf8]{inputenc}
\usepackage{amsmath}
\usepackage{amsfonts}
\usepackage{amssymb}
\usepackage{url}
\usepackage{graphicx} 
\usepackage{float}
\usepackage{enumitem}
\usepackage{tabularx}
\usepackage{booktabs} % Required for nicer horizontal rules in tables
\usepackage[spanish, es-tabla, es-nodecimaldot]{babel}		%Separacion silabica espanola
\usepackage[left=2.00cm, right=2.00cm, top=2.00cm, bottom=2.00cm]{geometry}
\usepackage{multicol}	% Para manejar columnas múltiples
\usepackage{longtable}  % Para manejar tablas de varias paginas con encabezado
		\newenvironment{Table}
		{\par\medskip\noindent\minipage{\linewidth}}
		{\endminipage\par\medskip}
\usepackage{fancyhdr}	% Para manejar los encabezados y pies de pagina
		\pagestyle{fancy}		% Contenido de los encabezados y pies de pagina
\usepackage{wrapfig}	% Necesario para la rubrica de evaluación
%--------Encabezado y pie de página---------------------
\lhead{Computacional I}
\chead{}
\rhead{Actividad 1}	% va el numero de experimento, al igual que en el titulo
\lfoot{Lic. en Física}
\cfoot{\thepage\ }
\rfoot{Universidad de Sonora}
%-----------Portada---------------------------------------------
\author{
Leonardo Coronado Arvayo\\
Profesor: Carlos Lizárraga Celaya   \vspace*{1.25in}}
\title{	\includegraphics[width=3cm]{Logo} \\
Universidad de Sonora \\
{\small Departamento de Física \\
Licenciatura en Física \\
Computacional I \\
2017-1 \\
\vspace{0.55in} Reporte}\\ 
{\Huge Actividad \#1.1: Preguntas de reflexión}\\
\vspace*{1.0in}}
%-----------Reporte----------------------------
%	Portada y tabla de contenidos
\begin{document}
	\pagenumbering{gobble} % Remove page numbers (and reset to 1)
	\maketitle
\newpage
	\pagenumbering{arabic}


\section{Preguntas}

\begin{itemize}
\item     ¿Cual es tu primera impresión de uso de LaTeX? es complicado de manejar, incluso usando plantillas ya diseñadas pero el resultado es muy presentable y es fácil de repetir (reutilizando la plantilla).
\item     ¿Qué aspectos te gustaron más? Me divierte aprender a usarlo sobre una plantilla ya hecha, adaptarla para mis propósitos es interesante. 
\item     ¿Qué no pudiste hacer en LaTeX? No se muy bien como cambiar la letra en secciones en especifico, como hacerla negrita o cursiva pero tampoco he necesitado hacerlo.
\item     En tu experiencia, comparado con otros editores, ¿cómo se compara LaTeX? Es mucho mas ordenado. Tambien pienso que es mas bonito el resultado (pero solo he utilizado Word y Office).
\item     ¿Qué es lo que mas te llamó la atención en el desarrollo de esta actividad? los indicadores de temperatura y humedad, también la pagina de la Universidad de Wyoming.
\item     ¿Qué cambiarías en esta actividad? No lo se.
\item     ¿Que consideras que falta en esta actividad? Creo que esta bien para comenzar a usar el editor. 
\item   ¿Puedes compartir alguna referencia nueva que consideras util y no se haya contemplado? no pude encontrar ninguna !
\item     ¿Algún comentario adicional que desees compartir? no de momento.
\end{itemize}
   



\end{document}