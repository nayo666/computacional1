%---------Definición de paquetes base--------------
\documentclass[12pt,letterpaper]{article}
\usepackage[utf8]{inputenc}
\usepackage{amsmath}
\usepackage{amsfonts}
\usepackage{amssymb}
\usepackage{url}
\usepackage{graphicx} 
\usepackage{float}
\usepackage{enumitem}
\usepackage{tabularx}
\usepackage{booktabs} % Required for nicer horizontal rules in tables
\usepackage[spanish, es-tabla, es-nodecimaldot]{babel}		%Separacion silabica espanola
\usepackage[left=2.00cm, right=2.00cm, top=2.00cm, bottom=2.00cm]{geometry}
\usepackage{multicol}	% Para manejar columnas múltiples
\usepackage{longtable}  % Para manejar tablas de varias paginas con encabezado
		\newenvironment{Table}
		{\par\medskip\noindent\minipage{\linewidth}}
		{\endminipage\par\medskip}
\usepackage{fancyhdr}	% Para manejar los encabezados y pies de pagina
		\pagestyle{fancy}		% Contenido de los encabezados y pies de pagina
\usepackage{wrapfig}	% Necesario para la rubrica de evaluación
%--------Encabezado y pie de página---------------------
\lhead{Computacional I}
\chead{}
\rhead{Actividad 2}	% va el numero de experimento, al igual que en el titulo
\lfoot{Lic. en Física}
\cfoot{\thepage\ }
\rfoot{Universidad de Sonora}
%-----------Portada---------------------------------------------
\author{
Leonardo Coronado Arvayo\\
Profesor: Carlos Lizárraga Celaya   \vspace*{1.25in}}
\title{	\includegraphics[width=3cm]{Logo} \\
Universidad de Sonora \\
{\small Departamento de Física \\
Licenciatura en Física \\
Computacional I \\
2017-1 \\
\vspace{0.55in} Reporte}\\ 
{\Huge Actividad \#2}\\
\vspace*{1.0in}}
%-----------Reporte----------------------------
%	Portada y tabla de contenidos
\begin{document}
	\pagenumbering{gobble} % Remove page numbers (and reset to 1)
	\maketitle
\newpage
	\pagenumbering{arabic}
	\tableofcontents
\pagebreak
%====================================================================================================
%=================================================Resumen==========================================
%====================================================================================================

\begin{abstract}
El presente reporte consta de dos partes. En la primera se comparan datos que relacionan la altura y temperatura obtenidos para La Paz, Baja California para el día 01 de febrero 2017 contra como es esperado estos comportamientos, cabe destacar que estos datos solo alcanzan 30 mil metros de altura. En la segunda parte se reporta como se usó Linux para obtener, compilar y revisar estos datos para el año 2016, estos datos se expiden diariamente. \\
\end{abstract}


%====================================================================================================
%=================================================Introducción=======================================
%====================================================================================================
\section{Introducción}

La presente practica tiene como objetivos en el área de la computación el uso de gnuplot para graficar, emacs para manejar y guardar información y Linux para manejar y copilar datos. Estas herramientas son importantes ya que son las requeridas para obtener, procesar y observar comportamientos de datos presentados en distintas formas. \\

Para esto el trabajo se separó en partes, en una primera se obtuvieron los datos de altura, presión y temperatura para la Paz para el día 01 de febrero del presente año\cite{a}. Estos datos se graficaron presión contra altura, y temperatura contra altura; usando gnuplot. Luego, usando la información presentada en la actividad anterior se compararon dichas observaciones con el comportamiento esperado de la temperatura y presión en relación a la altura. \\

La segunda parte consiste en la obtención de datos disponibles para la Paz en el 2016 usando un script proporcionado para esta misma actividad. Luego sigue compilar todos los datos en un archivo y revisar sus características, ya que aunque las mediciones se hacen diariamente, el equipo para realizarlas es costoso por lo que se espera que no siempre se realicen.\\

El trabajo se divide en un desarrollo donde se incluyen las dos partes mencionadas anteriormente, seguido de unas breves conclusiones.\\


%====================================================================================================
%============================================Desarrollo==============================================
%====================================================================================================
\section{Desarrollo}


\subsection{Propiedades de la atmósfera: concepción y evidencia empírica}

Primero se realiza un breve repaso de las características de la atmósfera para luego pasar a compararla con evidencia empírica obtenida para la Paz, Baja California para el día 01 de febrero del 2017. \\

La atmósfera tiene 4 capas que son\cite{a}:

\begin{itemize}
\item Troposfera: Es la capa mas baja de la atmósfera donde los seres vivos terrestres viven y donde se da el clima. La temperatura de esta capa disminuye al aumentar la altura. 
\item La frontera entre la troposfera y estratosfera se llama tropopausa. 
\item Estratosfera: esta entre los 30 y 50 km de altitud, a diferencia de en la troposfera en esta capa la temperatura aumenta junto con la altura; ya que en esta se encuentra la capa de ozono que tiene la propiedad de absorber los rayos ultravioletas del sol.
\item Mesosfera: Esta capa es similar a la troposfera ya que la temperatura disminuye con la altura y contiene proporciones similares de nitrógeno y oxigeno, aunque estas son mil veces menores y hay poco vapor de agua. Por este ultimo punto, el aire es demasiado delgado, lo cual no permite la presencia del clima.
\item  Termosfera o ionosfera: es la capa mas alta de la atmosfera, en esta capa la temperatura se incrementa con la altura ya que esta es directamente calentada por el sol.
\end{itemize}

En la siguiente imagen se pueden observar las capas de la atmósfera:

\begin{figure}[H]
	\centering
	\includegraphics[height=10cm]{atm}
	\caption{Capas de la atmósfera \cite{a}}
\end{figure}

Así mismo, la figura 1 muestra los cambios de la temperatura contra la altura, se puede observar como el comportamiento es lineal principalmente en la troposfera y cambia no linealmente conforme aumenta la altura. Esto sucede ya que las moléculas que conforman la atmósfera son mas vulnerables a la fuerza de gravedad conforme están mas cerca de la tierra, lo que significa que esta se concentra mas en la superficie terrestre y se hace mas delgada conforme aumenta la altura\cite{a}. Lo anterior a su ves significa que a mayor altura hay menor cantidad de moléculas sobre ti, lo que resulta en una menor presión\cite{a}. La siguiente imagen muestra la relación presión-altura:\\

\begin{figure}[H]
	\centering
	\includegraphics[height=10cm]{pres}
	\caption{Presión contra altura \cite{a}}
\end{figure}

\subsubsection{Comparación de mediciones con la realidad: Temperatura}
En la siguiente imagen se muestra la relación de temperatura contra altura para la Paz, Baja California. Se puede observar como la altura solo lleva a los 30 mil metros, por lo que estas mediciones de acuerdo a la imagen mostrada anteriormente deberían solo llegar hasta la estratosfera. \\

\begin{figure}[H]
	\centering
	\includegraphics[height=10cm]{TvsH1}
	\caption{Temperatura contra altura en La Paz, Baja California. Datos de\cite{a}}
\end{figure}


\begin{figure}[H]
	\centering
	\includegraphics[height=10cm]{comp1}
	\caption{Comparación de la imágenes de temperatura contra altura\cite{a}}
\end{figure}

Sin embargo, por lo que se puede observar en la figura 4, parece que el comportamiento presentado para la Paz es no llegando a la estratosfera, sino que parece incluirla. De forma similar el comportamiento estático de la temperatura en la troposfera y la capa con ozono máximo no es tan marcada como se plantea originalmente.  Obviamente estas son observaciones para un solo día, por lo que no necesariamente son representativas del comportamiento general de la atmósfera. Pero igualmente son resultados interesantes.

\subsubsection{Comparación de mediciones con la realidad: Presión}

Para el caso de la presión el comportamiento parece estar más concorde al de la esperada. Este comportamiento se puede observar en la figura 5. El comportamiento parece presentar un comportamiento cuadrático/exponencial bastante marco o bien, con pocas irregularidades. Al igual que en la relación anterior, solo se alcanza 30 mil metros de altura.\\

\begin{figure}[H]
	\centering
	\includegraphics[height=10cm]{PvsH}
	\caption{Temperatura contra altura en La Paz, Baja California. Datos de\cite{a}}
\end{figure}

Por último se muestra la comparación entre lo esperado y los datos obtenidos (véase la figura 6). Como se ve los comportamientos son muy similares (al menos en forma si no totalmente en la pendiente).  Parece que la forma asintótica en los datos empíricos es menos pronunciado, e igualmente se puede decir que la continuidad de estos datos no es perfecta. \\

\begin{figure}[H]
	\centering
	\includegraphics[height=10cm]{comp2}
	\caption{Comparación de mediciones con la realidad: Presión\cite{a}}
\end{figure}

\subsection{Obtención de datos para un año e irregularidades}

En esta subsección se describe la obtención de los datos para diarios de altura, temperatura, presión, humedad relativa, entre otros (que se emiten “diariamente”) para todo el año del 2016. \\

Primero se corrió el script proporcionado que baja los datos por mes. El mismo que solo requirió de ajustes menores (cambiar la ubicación del programa “wget” y cambiar el número de la estación para que fuera La Paz). Esto resulto en 12 archivos, uno para cada mes que contienen los datos diarios mencionados anteriormente. A continuacion se puede ver un fragmento de dicho spript:\\

\begin{verbatim}  

for i in $LISTM31 ; do

    /usr/bin/wget "http://weather.uwyo.edu/cgi-bin/sounding?region=naconf&TYPE=TEXT%3ALIST&YEAR=2016&MONTH=$i&FROM=0100&TO=3112&STNM=76405"

       /bin/sleep 5

done

# Months 30 days

for i in $LISTM30 ; do

    /usr/bin/wget "http://weather.uwyo.edu/cgi-bin/sounding?region=naconf&TYPE=TEXT%3ALIST&YEAR=2016&MONTH=$i&FROM=0100&TO=3012&STNM=76405"

       /bin/sleep 5

\end{verbatim}  

Una vez obtenidos, se utilizó el comando “cat” seguido del nombre conjunto de los archivos (son similares hasta cierto punto, luego se le indica a Linux que complete lo restante para cada uno), y se compilo en un archivo que se denominó “sondeos.txt”.x \\

Seguido comenzó el proceso de análisis de la información, se encontró lo siguiente:\\
\begin{itemize}
\item El archivo “sondeos.txt” contiene 29 094 renglones, 273 125 palabras y 2 078 045 caracteres (se encontró usando el comando “wc”).
\item Se buscaron las observaciones para las 12 horas Greenwich, usando una combinación de los comandos grep y cat, y se encontraron 198 líneas (u observaciones para esa hora).  
\item Se realizó el mismo procedimiento para las 00 horas Greenwich, y solo se encontrar 5 observaciones.
\end{itemize}

Seguido se analizaron los datos para cada mes y los resultados se pueden observar en la siguiente tabla:\\

\begin{table}[ht]
\begin{center}
\begin{tabular}{|c|m{10cm}|}
\hline
Mes &  Mediciones encontradas (comentarios) \\
\hline
Enero	&	Para enero se encontró que hubo mediciones para 22 días (se usó el comando cat, wc y grep para esto) \\
Febrero	&	En febrero se tienen 21 observaciones	\\
Marzo	&	Para marzo son 14	\\
Abril 	&	Abril tiene 29 (ósea que casi todos los días se mandó un globo)	\\
Mayo	&	En mayo se encontraron 16	\\
Junio	&	Junio 28	\\
Julio	&	Julio 26 	\\
Agosto	&	Agosto solo 8	\\
Septiembre	&	Septiembre 15	\\
Octubre 	&	Para octubre no se encontró ninguna	\\
Noviembre	&	Noviembre también tiene 0 observaciones	\\
Diciembre	&	Diciembre 24	\\

\hline
\end{tabular}
\end{center}
\caption{Observaciones encontradas por mes}
\end{table}

Cabe destacar que para la obtención de las mediciones se requiere soltar un globo meteorológico, por lo que es lógico suponer que no siempre es posible realizarlo.

%====================================================================================================
%================================================Conclusiones==========================================
%====================================================================================================
\section{Conclusiones} 

El uso de scripts en Linux para la obtención datos resulto se bastante efectivo. Además, los comandos para revisarlos parecen claros y fáciles de replicar. En general se consideran herramientas muy útiles y necesarias de conocer.

%====================================================================================================
%===============================================Referencias==========================================
%====================================================================================================
\begin{thebibliography}{2}
\bibitem{a} \url{http://climate.ncsu.edu/edu/k12/.AtmStructure}

\end{thebibliography}


\end{document}