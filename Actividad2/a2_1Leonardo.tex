%---------Definición de paquetes base--------------
\documentclass[12pt,letterpaper]{article}
\usepackage[utf8]{inputenc}
\usepackage{amsmath}
\usepackage{amsfonts}
\usepackage{amssymb}
\usepackage{url}
\usepackage{graphicx} 
\usepackage{float}
\usepackage{enumitem}
\usepackage{tabularx}
\usepackage{booktabs} % Required for nicer horizontal rules in tables
\usepackage[spanish, es-tabla, es-nodecimaldot]{babel}		%Separacion silabica espanola
\usepackage[left=2.00cm, right=2.00cm, top=2.00cm, bottom=2.00cm]{geometry}
\usepackage{multicol}	% Para manejar columnas múltiples
\usepackage{longtable}  % Para manejar tablas de varias paginas con encabezado
		\newenvironment{Table}
		{\par\medskip\noindent\minipage{\linewidth}}
		{\endminipage\par\medskip}
\usepackage{fancyhdr}	% Para manejar los encabezados y pies de pagina
		\pagestyle{fancy}		% Contenido de los encabezados y pies de pagina
\usepackage{wrapfig}	% Necesario para la rubrica de evaluación
%--------Encabezado y pie de página---------------------
\lhead{Computacional I}
\chead{}
\rhead{Actividad 2}	% va el numero de experimento, al igual que en el titulo
\lfoot{Lic. en Física}
\cfoot{\thepage\ }
\rfoot{Universidad de Sonora}
%-----------Portada---------------------------------------------
\author{
Leonardo Coronado Arvayo\\
Profesor: Carlos Lizárraga Celaya   \vspace*{1.25in}}
\title{	\includegraphics[width=3cm]{Logo} \\
Universidad de Sonora \\
{\small Departamento de Física \\
Licenciatura en Física \\
Computacional I \\
2017-1 \\
\vspace{0.55in} Reporte}\\ 
{\Huge Actividad \#2.1: Preguntas de reflexión}\\
\vspace*{1.0in}}
%-----------Reporte----------------------------
%	Portada y tabla de contenidos
\begin{document}
	\pagenumbering{gobble} % Remove page numbers (and reset to 1)
	\maketitle
\newpage
	\pagenumbering{arabic}


\section{Preguntas}

\begin{itemize}

\item ¿Cual es tu primera impresión del uso de bash/Emacs? Quisiera que reconociera los comandos ctrl + z, ctrl + c, ctrl + v. Fuera de eso me resulto agradable. Como un blog de notas más sofisticado.
\item ¿Ya lo habías utilizado? Si, para programar en fortran. 
\item ¿Qué cosas se te dificultaron más en bash/Emacs?  Nada en específico, aunque tiendo a olvidar los comandos. Pero la interfaz gráfica ayuda.
\item ¿Qué ventajas les ves a Emacs? Me gusta que reconozca los comandos cuando programas (por colores), es bastante útil.
\item ¿Qué es lo que más te llamó la atención en el desarrollo de esta actividad? El uso de los scrpts de Linux, fue bastante interesante su uso.
\item ¿Qué cambiarías en esta actividad? Explicar los comandos de script usando $\#$, para saber exactamente que hace cada cosa y con qué palabra.
\item ¿Que consideras que falta en esta actividad? Me hubiera gustado poder homogenizar los datos del 2016 (quitar los excesos en las tablas).
\item ¿Puedes compartir alguna referencia nueva que consideras útil y no se haya contemplado? No creo haber encontrado ninguna.
\item ¿Algún comentario adicional que desees compartir? No de momento.
 

\end{itemize}
   



\end{document}